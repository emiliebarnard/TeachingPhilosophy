% LaTeX

\documentclass[12pt]{amsart} \usepackage{amssymb}

%\textwidth = 460pt 
%\textheight = 9in 
%\hoffset=-54pt \voffset=-40pt

% SIDE MARGINS:
\oddsidemargin 0in \evensidemargin 0in

% VERTICAL SPACING:
%\topmargin -.15in
\topmargin -.5in
\headheight 0in \headsep 0.0in
%\footheight 0.5in
\footskip 0.5in

\pagestyle{plain}
%\pagenumbering{}

% DIMENSION OF TEXT:
\textheight 10in \textwidth 6.7in
%

%\textwidth = 470pt
%\textheight = 700pt
%%\topmargin = 0pt
%%\oddsidemargin = 0pt
%\hoffset = -60pt
%\voffset = -50pt

%\input epsf \def\epsfsize#1#2{0.4#1\relax} \def\nl{\hfil\break}

%\renewcommand{\baselinestretch}{1.2}
%\def\Indent{\hskip .2in}

\def\labelitemi{--}

\title[]{Statement of Teaching Philosophy}

\author[]{Emilie Menard Barnard}

\begin{document}
\maketitle
\thispagestyle{empty}

%\section*{Background}
%\noindent I knew nothing about computer science before I started university. After taking a few core courses, I started to realize some disadvantages I had as a student in the computer science program. Many of my peers' parents not only had college degrees, but also held degrees in computer science or related fields. They were able to discuss their projects and new course concepts learned with their families. I, however, have not been able to discuss my courses with my parents since the 5th grade. Neither of my parents have ever programmed a computer, or even attended college. Gender differences set me further apart from my peers. I will always remember the discrimination I faced in one of my sophomore-year courses. My project group was the first female-only group our professor had seen, which seemed to give the professor and my peers doubts that our group could complete our project. I was shocked by the comments in class, and can only imagine the content of the comments that I did not hear.

\section*{Why I Teach}
\noindent My goal as a computer science educator is to craft a welcoming and productive environment for my students to explore the field. \textbf{I want everyone to feel like they can study and learn computer science if they are interested, no matter their background or previous experience.} I firmly believe that all students can excel in the classroom (and beyond) with an effective teaching style, proper encouragement, and enough self-motivation.

\section*{How I Teach}
\noindent As an educator, I strive to:
\begin{itemize}
\item \emph{Simplify}: Computer science is presented as a difficult field. Before some students even enroll in their first course, they hold preconceived ideas of how daunting the material will be. I start destroying these concerns on the first day of class. I focus less on flashy ``buzz words" and instead break down material into small chunks of information that are then pieced together like a fun logic puzzle.
\vspace{2mm}
\item \emph{Provide transparency}: By sharing my reasons for a lesson (both the content itself and how it is presented), students feel more involved with their learning process. I usually find that this ``humanizes" their experience and helps students stay more engaged.
\vspace{2mm}
\item \emph{Encourage autonomy}: This is extremely important to teach in introductory courses so that students are better prepared for advanced coursework and employment. For instance, when a student asks a coding question that can easily be found in language specifications, I show them how they can find the answer online. By doing so, I help them learn how to help themselves.\vspace{2mm}
\item \emph{Be approachable}: Students need to feel comfortable approaching me with questions, ideas, or even just to discuss general topics. This fosters a better learning environment by encouraging questions and strengthening the relationships with my students. Mentoring is one of my favorite aspects of teaching, so this is very important to me.\vspace{2mm}
\item \emph{Foster an inclusive learning environment}: Students often feel intimidated by their peers or coursework. I frequently remind them that it is normal to feel this way, especially in a discipline like computer science where impostor syndrome runs deep, and that I myself even feel this way sometimes. I encourage them to not give up and remind them that what matters is what \emph{they} get out of the class, not how others appear to be progressing. \\As a woman in computer science, I have experienced first-hand how strong of an impact microaggressions and other nonsensical remarks can have on one's learning process. As such, I have a zero-tolerance policy for any form of discrimination in my classroom and encourage open and respectful discussions on the topic when appropriate. \vspace{2mm}
\end{itemize}
In short, my hope is that my students feel comfortable and confident so that they can learn as much as possible. I want \emph{every} student to be proud of their learning experience and excited to learn more about computer science.

%\section*{My Current Goals for Growth}
%\noindent I am looking forward to accepting the responsibilities of an official course instructor again. I am hopeful that my fresh perspective will improve each course I am able to teach, and am excited to see how my approach helps students.
%\\\\\noindent I am also interested in teaching more varied courses. I currently have significant experience teaching introductory courses using the Python programming language, but I am eager to teach other types of courses as well.
%\\\\\noindent In addition, I hope to support and encourage students outside of the classroom. I am interested in service positions in addition to lectureship appointments.
%
%\section*{Conclusion}
%\noindent I am passionate about academics, schooling, and the learning process. I am excited to continue to give back, especially in more official educator roles. Most importantly, I want to be an example for the many students who, like myself, have been in disadvantaged and isolated positions. I want them to know that they too can pursue higher education and succeed in their academics, especially in a discipline as demanding as computer science.
\end{document}
