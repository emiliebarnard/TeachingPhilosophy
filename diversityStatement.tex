% LaTeX

\documentclass[12pt]{amsart} \usepackage{amssymb}

%\textwidth = 460pt 
%\textheight = 9in 
%\hoffset=-54pt \voffset=-40pt

% SIDE MARGINS:
\oddsidemargin 0in \evensidemargin 0in

% VERTICAL SPACING:
%\topmargin -.15in
\topmargin -.5in
\headheight 0in \headsep 0.0in
%\footheight 0.5in
\footskip 0.5in

\pagestyle{plain}
%\pagenumbering{}

% DIMENSION OF TEXT:
\textheight 10in \textwidth 6.7in
%

%\textwidth = 470pt
%\textheight = 700pt
%%\topmargin = 0pt
%%\oddsidemargin = 0pt
%\hoffset = -60pt
%\voffset = -50pt

%\input epsf \def\epsfsize#1#2{0.4#1\relax} \def\nl{\hfil\break}

%\renewcommand{\baselinestretch}{1.2}
%\def\Indent{\hskip .2in}

\def\labelitemi{--}

\title[]{Diversity Statement}

\author[]{Emilie Menard Barnard}

\begin{document}
\maketitle
\thispagestyle{empty}


%What is your definition of diversity? How do you encourage diverse student learners?
%
%
%start with some anecdote
%dive into my definition of diversity and why i think its important
%
%then talk about how do i encourage diversity


Only 4\% of students in introductory computer science courses in 2015 were first-generation, female college students.\footnote{http://dl.acm.org/citation.cfm?id=3017751} I learned this while attending a UCLA BRAID research group panel at SIGCSE 2017, my first computer science education conference. Their explores the relationship between these students' experiences in these courses and their feelings of self-efficacy and belonging in the computer science environment. 

I worked very hard to become a first-generation college graduate, and am proud to say I am a female with a Masters degree in Computer Science. One of my favorite aspects of my job is teaching and inspiring students who, like me, come from diverse backgrounds. As an educator I have two goals: to teach computer science, and for \emph{all of my students to feel confident enough to continue pursuing the field}. I do not want any of my students to be afraid of learning computer science.

Today, computer science affects all aspects of life. Code is everywhere - it's in our homes, our workplaces, even our grocery stores. All types of people interact with code, so it only makes sense that all types of people should be designing and creating this code. It is imperative that we encourage \textbf{diversity} --- people with different interests and preferences, and various cultural, economic, and ethnic backgrounds --- in computer science. \emph{Our world is diverse; diversity is the norm. It as abnormal that our field is currently not diverse.}

This philosophy is foundational to how I teach. To encourage diverse student learners, I do the following:

\begin{itemize}
\item \emph{Use gender-neutral pronouns} while addressing the class. Admittedly, my default pronoun for a group of people is ``you guys." To avoid gender-bias, I am working on switching this to ``y'all."  \vspace{2mm}
\item \emph{Avoid using names in examples}. Instead of using names that may hold cultural or gender biases, I use animals in my examples. If I absolutely must use a person in my example, I use terms like Person A, Person B, etc.
\vspace{2mm}
\item \emph{Make expectations for assignments and the course as clear as possible}. By stating explicit policies in my syllabus, I avoid any assumptions regarding what my students may be used to in a classroom environment. This also ensures every student is held to the same standards.
\vspace{2mm}
\item \emph{Assign a diverse set of projects}. When creating my curriculum, I design a diverse set of projects in an attempt to cover as many different interests as possible by the end of the term. My courses usually include graphical and artistic projects, text-based programs, data-mining projects using real data from NASA, and projects in cryptography.
\vspace{2mm}
\item \emph{Listen to my students} when they talk about course content and general experiences. I also spend some time during our first class meeting to introduce impostor syndrome to encourage open classroom discussion.
\vspace{2mm}
\item \emph{Reflect on my own biases}. I acknowledge that I am human which means I have biases. To be a better educator, I take time to consider my own biases to ensure I am doing the best I can. After each class meeting I take some time to think about which individual students I spent the most time helping, for example.
\end{itemize}

I am honored to be part of a student's first introduction to computer science. It is an incredible personal success when a student tells me at the end of the class, ``I didn't think computer science was for me, but after taking this class I am now seriously considering how I can incorporate it into my life and career path." I firmly believe that encouraging diversity in introductory courses is how we start changing the field of computer science for the better.
\end{document}
