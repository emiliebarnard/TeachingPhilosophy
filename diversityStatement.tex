% LaTeX

\documentclass[12pt]{amsart} \usepackage{amssymb}

%\textwidth = 460pt 
%\textheight = 9in 
%\hoffset=-54pt \voffset=-40pt

% SIDE MARGINS:
\oddsidemargin 0in \evensidemargin 0in

% VERTICAL SPACING:
%\topmargin -.15in
\topmargin -.5in
\headheight 0in \headsep 0.0in
%\footheight 0.5in
\footskip 0.5in

\pagestyle{plain}
%\pagenumbering{}

% DIMENSION OF TEXT:
\textheight 10in \textwidth 6.7in
%

%\textwidth = 470pt
%\textheight = 700pt
%%\topmargin = 0pt
%%\oddsidemargin = 0pt
%\hoffset = -60pt
%\voffset = -50pt

%\input epsf \def\epsfsize#1#2{0.4#1\relax} \def\nl{\hfil\break}

%\renewcommand{\baselinestretch}{1.2}
%\def\Indent{\hskip .2in}

\def\labelitemi{--}

\title[]{Diversity Statement}

\author[]{Emilie Menard Barnard}

\begin{document}
\maketitle
\thispagestyle{empty}


%What is your definition of diversity? How do you encourage diverse student learners?
%
%
%start with some anecdote
%dive into my definition of diversity and why i think its important
%
%then talk about how do i encourage diversity


Only 4\% of students in introductory computer science courses in 2015 were first-generation, female college students.\footnote{http://dl.acm.org/citation.cfm?id=3017751} I learned this while attending a panel at SIGCSE 2017, my first computer science education conference. This paper from UCLA's BRAID research group explores the relationship between these students' experiences in these courses and their feelings of self-efficacy and belonging in the computer science environment. 

I worked very hard to become a first-generation college graduate, and am proud to say I am a female with a Masters degree in Computer Science. One of my favorite aspects of my job is teaching and inspiring students who, like me, come from diverse backgrounds. As an educator I have two goals: to teach computer science, and for \emph{all of my students to feel confident enough to continue pursuing the field if they desire}. I do not want any of my students to be afraid of learning the subject.

Today, computer science affects all aspects of life. Code is everywhere - it's in our homes, our workplaces, our grocery stores, etcetera. All types of people interact with code daily, so it only makes sense that all types of people should be designing and creating this code. It is imperative that we encourage \textbf{diversity} - people with different interests and preferences, and various cultural, economic and ethnic backgrounds - in computer science. \emph{Our world is diverse; diversity is the norm. It as abnormal that the field is currently not diverse.}

This philosophy is foundational to how I teach.

%gender neutral programs when addressing the class (y'all is a favorite of mine)

%avoid using names in examples - instead use names of animals or if i must person A, person B, etc. to avoid any cultural bias

%make expectations for the course and assignmenets as explicit as possible to avoid any assumptions about what my students are used to in a classroom environment and outline clear policies to ensure all students are held to the same standards


% When creating my cirriculum, I design a diverse set of projects in an attempt to hit as many different interests as possible: graphical programs for the artistcally inclined, text-based projets for those who enjoy writing, data-driven projects using real data from NASA for those into science, and projects in cryptography for those who enjoy mathematics

%listen to my students both when they talk about course content and when they talk about experiences. To help set the tone for the classroom, I also spend some time during my first lecture talking about impostor syndrome.

%In the end, I am human so I have biases. To be a better educator, I take time to reflect on my own biases to ensure i am doing the best i can. Example - am I spending more time with certain students during break out activities?

\vspace{5cm}
I am honored to be there for students trying out computer science for the first time. It is an incredible personal success when a student tells me at the end of the class ``I didn't think computer science was for me, but after taking this class I am now seriously considering how I can incorporate it into my life and career path." Encouraging diversity in introductory courses is how we start changing the field of computer science for the better.




%\section*{Background}
%\noindent I knew nothing about computer science before I started university. After taking a few core courses, I started to realize some disadvantages I had as a student in the computer science program. Many of my peers' parents not only had college degrees, but also held degrees in computer science or related fields. They were able to discuss their projects and new course concepts learned with their families. I, however, have not been able to discuss my courses with my parents since the 5th grade. Neither of my parents have ever programmed a computer, or even attended college. Gender differences set me further apart from my peers. I will always remember the discrimination I faced in one of my sophomore-year courses. My project group was the first female-only group our professor had seen, which seemed to give the professor and my peers doubts that our group could complete our project. I was shocked by the comments in class, and can only imagine the content of the comments that I did not hear.

%\section*{Why I Teach}
%\noindent My goal as a computer science educator is to craft a welcoming and productive environment for my students to explore the field. \textbf{I want everyone to feel like they can study and learn computer science if they are interested, no matter their background or previous experience.} I firmly believe that all students can excel in the classroom (and beyond) with an effective teaching style, proper encouragement, and enough self-motivation.


%\section*{My Current Goals for Growth}
%\noindent I am looking forward to accepting the responsibilities of an official course instructor again. I am hopeful that my fresh perspective will improve each course I am able to teach, and am excited to see how my approach helps students.
%\\\\\noindent I am also interested in teaching more varied courses. I currently have significant experience teaching introductory courses using the Python programming language, but I am eager to teach other types of courses as well.
%\\\\\noindent In addition, I hope to support and encourage students outside of the classroom. I am interested in service positions in addition to lectureship appointments.
%
%\section*{Conclusion}
%\noindent I am passionate about academics, schooling, and the learning process. I am excited to continue to give back, especially in more official educator roles. Most importantly, I want to be an example for the many students who, like myself, have been in disadvantaged and isolated positions. I want them to know that they too can pursue higher education and succeed in their academics, especially in a discipline as demanding as computer science.
\end{document}
